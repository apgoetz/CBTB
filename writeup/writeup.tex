\documentclass[twocolumn]{article}
\usepackage{graphicx}
\usepackage{amsmath}
\usepackage{siunitx}
\usepackage{fancyhdr} 
\usepackage{fancybox}
\usepackage{float}
\usepackage{listings}
\usepackage[colorlinks=true,linkcolor=black]{hyperref}
%\usepackage[labelformat=empty]{caption}
\usepackage[margin=1.0in]{geometry}
\usepackage[section]{placeins}
\pagestyle{fancy}

%redefines subsections with letters instesad of numbers
\renewcommand{\thesubsection}{\thesection.\alph{subsection}}

% Center Image Command
\newcommand{\centerimage}[3]{
\begin{figure}[ht!]  
\begin{center}
#1
\caption{#2}
\label{#3}
\end{center}
\end{figure}}

\newcommand{\tstamp}{\today}   
%\renewcommand{\chaptermark}[1]{\markboth{#1}{}}
\renewcommand{\sectionmark}[1]{\markright{#1}}
\lhead[\fancyplain{}{\thepage}]         {\fancyplain{}{Andy Goetz \& Kevin Riedl}}
\chead[\fancyplain{}{}]                 {\fancyplain{}{}}
\rhead[\fancyplain{}{\rightmark}]       {\fancyplain{}{ECE 486 Branch Target Buffer Predictor \& Alpha Predictor}}
\lfoot[\fancyplain{}{}]                 {\fancyplain{\tstamp}{\tstamp}}
\cfoot[\fancyplain{\thepage}{}]         {\fancyplain{}{}}
\rfoot[\fancyplain{\tstamp} {\tstamp}]  {\fancyplain{}{\thepage}}

\author{\LARGE Andy Goetz \& Kevin Riedl}
\date{\today}
\title{\Huge \textbf{ECE 486 Branch Target Buffer \& Alpha Predictor}}

\begin{document}
\maketitle

\section{Abstract}
\section{Acknowledgements}

The authors would like to thank Tyler Tricker, Bradon Kanyid, and Eric
Krause for the many hours of fruitful discussions. Additionally, they
would like to thank Beth Krause for the delicious cookies provided.

\section{Background Information}
The purpose of this project was to develop a branch predictor for an
unknown ISA. We were required to copy the branch predictor used in the
Alpha 21264 processor. We were then required to design a branch target
predictor, with the requirement that it use only 8 kilobytes of
state. This branch predictor was then tested against an array of 20
test traces, in order to determine its performance. 

\section{Implementation Details}
\subsection{Alpha Predictor}
The branch predictor used in this projectwas modeled after
The one described in R. E. Kessler's paper on the Alpha 
21264 processor.
%ADD STUFFS ABOUT ALPHA PREDICTOR CHOICES NOT IN PAPER!!!
\subsection{Implementation Details Not Specified}
%initialization values of the tables
The Alpha paper did not specify a few details about how the predictor
works. The biggest missing piece was the initialization values of the tables.
Over many branches, the values do not matter, but for a small data set given



\subsection{Branch Target Predictor} 

In order to be truly successful, a branch predictor must be carefully
tuned to its target workload. The fact that the set of all traces our
branch predictor would be tested against was known ahead of time
presented us with an unbridled opportunity for benchmarketing,
however, we still needed to gather information about the traces, in
order to design an effective branch target predictor. 

Figure \ref{ddelta} shows the ECDF of branch displacement for the
first four benchmarks. As can be seen from this graph, virtually all
of the data.

\centerimage{\includegraphics[width=\columnwidth]{ecdf-serv-delta.pdf}}{CDF
  of Branch Displacement Size}{ddelta}


The branch target predictor used in the simulator can be seen in
figure \ref{btbshape}. It is based on a hierarchy of two separate
caches, combined with a return address stack. A small displacement
cache (64 entries by 2 ways) holds entries whose target address is
less than 128 bytes from the program counter. If a target address has
a displacement that is too large to fit in this cache, or an address
is evicted from the displacement cache, it is placed in another,
larger cache that contains direct-mapped addresses. This larger cache
is organized as 64 entries by 8 ways.

\centerimage{\includegraphics[width=\columnwidth]{BTB.png}}{Branch
  Target Predictor}{btbshape}

The branch target predictor also contains a return address stack. This
stack stores the return address of the last 21 calls. This allows
return addresses to be predicted, regardless of whether or not they
are in the cache. In addition to storing return addresses in the RAS,
return addresses are also placed in the displacement and direct
caches.

In order to determine the optimal branch predictor design, a generic
predictor framework was designed that used environment variables to
determine the cache hierarchy used by the branch target predictor.
The tunable parameters can be seen in table \ref{envars}. 


\begin{table}
\begin{center}\begin{tabular}{p{.35\columnwidth}p{.5\columnwidth}}
Variable Name & Description \\
\hline
\texttt{BTB\_MAIN\_SIZE} & Index bits of direct cache \\
\texttt{BTB\_MAIN\_WAYS} & Number of ways of direct cache \\
\texttt{BTB\_DISP\_ENTRIES} & Size of entry for displacement cache \\
\texttt{BTB\_DISP\_SIZE} & Index bits of displacement cache \\
\texttt{BTB\_DISP\_WAYS} & Number of ways of displacement cache \\
\texttt{BTB\_WAY\_ALGO} & Way Eviction Policy (LRU or Round Robin) \\
\texttt{BTB\_FUNC\_CAP} & Number of entries in RAS pp
\end{tabular}\end{center}
\caption{BTB Tunable Parameters}
\label{envars}
\end{table}

In addition, a replacement predictor framework was written using
plaintext tracefiles, allowing for much faster traces, as well as more
detailed predictor statistics. These statistics included breakdown of
the percentage of misses that were caused by function calls,
conditional branches, and indirect branches, among other values. 

\section{Testing Methodology}
\centerimage{\includegraphics[width=\columnwidth]{random.pdf}}{Performance
  of Random Sized BTB Caches}{bgraph}
%Oracle predictor and BTB variables
%Table comparing Faust results to our results for Alpha
\section{Results}
%Table with all prediction rates for all tests
\section{Conclusion}
\newpage
\onecolumn
\section{Predictor.cc}
\lstinputlisting{../predictor.cc}[language=C++, showstringspaces=false]
\end{document}

